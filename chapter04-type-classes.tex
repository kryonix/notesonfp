%!TEX root = fp.tex

% Author: Philipp Moers <soziflip funny character gmail dot com>

\section{Type Classes} % (fold)
\label{cha:type_classes}

A \textbf{type class} C defines a family of type signatures (''methods'') which all \textbf{instances} of C must implement.

\begin{Haskell}
class C a where
    f@$_1$@ :: t@$_1$@
    @\dots@
    f@$_n$@ :: t@$_n$@
\end{Haskell}

The \codeline{t|$_i$|} \underline{must} mention a.\\
For any \codeline{f|$_i$|} the class may provide default implementations. \\
We have \codeline{f|$_i$| :: C a => t|$_i$|} \\ (read ''if a is instance C then f$_i$ has type t$_i$'').\\
\codeline{C a} is called \textbf{class constraint}.

\textit{Example}:

\begin{Haskell}
class Eq a where
    (==) :: a -> a -> Bool
    (/=) :: a -> a -> Bool
    x == y = not (x /= y)
    x /= y = not (x == y)
\end{Haskell}

(These are default implementations. To redefine one of them is sufficient.)

% \codefile{haskell}{caption={type-classes.hs}, label=type-classes.hs}{../material/type-classes.hs}


\subsection{Class Inheritance}

\begin{itemize}
    \item Defining \codeline{class (c|$_1$| a, c|$_2$| a, |\dots|) => C a where |\dots|} makes type class C a subclass of the C$_i$.
    \item \codeline{C a => t} implies \codeline{C|$_1$| a, C|$_2$| a |\dots|}.
\end{itemize}

% \vspace{9pt}
% \includegraphics[scale=1.0]{type-class-inheritance.png}
% \textcolor{myorange}{[Here is an image missing]}
% \vspace{9pt}


\subsection{Class Instances}

If type t implements the methods of class C, t becomes an \textbf{instance of} C:\\
\begin{Haskell}
instance C t where
    f@$_1$@ = <def of f@$_1$@>
    @\dots@
    f@$_n$@ = <def of f@$_n$@>
\end{Haskell}

(All defs of f$_i$ may be provided, minimal complete definition \underline{must} be provided.)
Class constraint C t is satisfied from now on.

\textit{Example}:\\
\begin{Haskell}
instance Eq Bool where
    x == y = x && y || (not x && not y)
\end{Haskell}

An instance definition for type constructor t may formulate class constraints for its argument types a, b, \dots:
\codeline{instance (C|$_1$| a, C|$_2$| a, |\dots|) => C t where}

\begin{mdframed}[linecolor=black, topline=false, bottomline=false,
  leftline=false, rightline=false, backgroundcolor=mintedbackground]
    \inputminted[fontfamily=tt]{Haskell}{../material/rock-paper-scissor-inst.hs}
    %\Haskellcode{../material/rock-paper-scissor-inst.hs}
\end{mdframed}

\subsubsection{Deriving Class Instances}
Automatically make user-defined data types (\codeline{data |\dots|}) instances of classes C$_i$ $\in$ \{ Eq, Ord, Enum, Bounded, Show, Read \}:

\begin{Haskell}
data T a@$_1$@ a@$_1$@ @\dots@ a@$_n$@ = @\dots@
                   | @\dots@
    deriving (C@$_1$@, C@$_2$@, @\dots@)
\end{Haskell}

\Haskellcode{../material/rock-paper-scissor-inst-deriving.hs_}

% chapter type_classes (end)
