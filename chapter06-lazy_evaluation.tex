%!TEX root = fp.tex

\section{Lazy Evaluation}
To execute a program, Haskell reduces expressions to values. Haskell uses normal order reduction to select the next expression to reduce:

\begin{enumerate}
   \item The outermost reducible expression (redex) is reduced first
   \item $\Rightarrow$ Function applications are reduced first before their arguments.
   \item If no further redex is found, the expression is in normal form and reduction terminates.
\end{enumerate}

\begin{Haskell}
fst :: (a, b) -> a
fst (x, y) = x
sqr :: Num a => a -> a
sqr x = x * x

-- ->: reduces to
fst (sqr (1+3), sqr 2) -> sqr (1 + 3) [fst]
                       -> (1+3) * (1+3) [sqr]
                       -> 4 * 4 [+/+]
                       -> 16 [*]
\end{Haskell}

Haskell avoids the duplication of work through graph reduction. Expression are shared (referenced more than once) instead of duplicated.\\
Lazy evaluation: normal order reduction and sharing.

\subsection{WHNF}
An expression e is in weak head normal form (WHNF) if it is of the following form
\begin{enumerate}
   \item v (where v is an atomic value \mint{Haskell}{Integer, Bool, Char, ...})
   \item \mint{Haskell}{c e1 e2 ... en} (where c is an n-ary constructor, like \mint{Haskell}{(:)})
   \item \mint{Haskell}{f e1 e2 ... em} (where f is an n-ary function, $m<n$)
\end{enumerate}

Haskell reduces values to WHNF only (stop criterium for reduction) unless we request reduction to normal form (e.g. when printing results).\\

Example expressions in WHNF:
\begin{itemize}
   \item \mint{Haskell}{42 --1.}
   \item \mint{Haskell}{(sqr 2, sqr 4) --2. (,)}
   \item \mint{Haskell}{f x : map f xs --2. (:)}
   \item \mint{Haskell}{Just (40+2) --2. Just}
   \item \mint{Haskell}{(* (40+2)) --3. * binär}
   \item \mint{Haskell}{(\x -> 40+2) --3. unary function w/o args}
\end{itemize}

\subsection{Lazy Evaluation and Bottom}
Some Haskell expressions have the value bottom. Examples: \mint{Haskell}{error, undefined, bomb}. Lazy evaluation admits functions that return a non-bottom value even if they receive bottom as argument (also: non-strict functions).
N-ary function f is strict in its i-th argument, if \mint{Haskell}{f x1 .. xi-1 bottom xi+1 ... xn = bottom}
Examples:
\begin{itemize}
   \item \mint{Haskell}{const :: a -> b -> a --strict in first, non-strict in second argument}
   \item \mint{Haskell}{(&&) :: Bool -> Bool -> Bool -- dito}
\end{itemize}
If a function pattern matches on an argument, Haskell semantics define it to be strict in that argument.\\
Example:\\
\begin{Haskell}
data T = T Int
f :: T -> Int
f (T x) = 42

f undefined -> undefined
f (T undefined) -> 42
\end{Haskell}


%14.01.2016

\section{Infinite Lists (Data Structures)}
One consequence of lazy evaluation: programs can handle infinite Lists as long as any run will inspect only a finite prefix of such a list.
Enables a modular programming style:
\begin{enumerate}
   \item generator functions produce an infinite number of solutions / approximations / \dots
   \item test functions select one (or finite number of) solutions from this infinite list.
\end{enumerate}

\subsection{Example: Newton-Raphson square root approximation}
Iteratively approximate the square root of x:
\begin{enumerate}
   \item $a_0 = \tfrac{x}{2}$
   \item $a_{i+1} = \tfrac{a_i+\tfrac{x}{a_i}}{2}$
\end{enumerate}

\Haskellcode{../material/newton-raphson.hs}

%\begin{Haskell}
%--1. Generator:
%iterate :: (a -> a) -> a -> [a]
%--iterate f x @\equiv@ [x, f x, f (f x), @\dots@]
%iterate f x = x : iterate x (f x)
%
%--2. Test:
%within :: (Ord a, Num a) => a -> [a] -> a
%--within eps xs : consumes xs until adjacent elements differ less than eps (for the first time)
%within eps (@x_1@:@x_2@:xs) | abs (@x_1@-@x_2@) <= eps = @x_2@
%                                                | otherwise                                = within eps (@x_2@:xs)
%\end{Haskell}

\subsection{Example: Tic-Tac-Toe game tree}
Build the (potentially huge) tree of possible moves for the Tic-Tac-Toe board game. Evaluate promise of game position. 
Plan:
\begin{enumerate}
   \item Find representation of game position (board + player next up)
\begin{verbatim}
|1|2|3| next: x
|4|5|6| square #6: open 
|0|x|0| spare #9: occupied by player 0
\end{verbatim}
   \item Provide pretty-printing for game positions.
   \item Define initial position and possible moves: \mint{Haskell}{moves :: Position -> [Position]}
   \item Evaluate a given position: \mint{Haskell}{static :: Position -> Int} (1 / -1: x/0 won the game, 0 draw)
\end{enumerate}

%\Haskellcode{../material/tic-tac-toe.hs}