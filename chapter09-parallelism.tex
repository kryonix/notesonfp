%!TEX root = fp.tex

% Author: Philipp Moers <soziflip funny character gmail dot com>


\chapter{Parallelism} % (fold)
\label{cha:parallelism}
(`//ism')


\section{semi-explicit parallelism (1)}

Combinator \codeline{par :: a -> b -> b} (module Control.Parallel) \codeline{e$_1$ `par` e$_2$}
indicates that it may be beneficial to evaluate \codeline{e$_1$} and \codeline{e$_2$} in parallel. Haskell runtime wraps evaluation of \codeline{e$_1$} in a \textbf{spark}, a a computation that \underline{may} be evaluated in a parallel thread.\\
Result is \codeline{e$_2$}. 

\codefile{haskell}{caption={parFibPie.hs}, label={parFibPie.hs}}{../material/parFibPie.hs_ASCII}


Compile Haskell program with runtime system (RTS)\\
\codeline{ghc -threaded-rtsopts -eventlog [--make] \textit{prog}.hs}\\
Specify RTS option and run:\\
\codeline{prog [options] +RTS -N\textit{n} -A\textit{size} -s -l -RTS}

\begin{itemize}
    \item \codeline{-N\textit{n}} run with \textit{n} parallel threads (e.g. number of CPUs in system)
    \item \codeline{-A\textit{size}} allocation area for garbage collector (default is \SI{512}{\kibi\byte})
    \item \codeline{-s} profiling summary
    \item \codeline{-l} write event log (e.g. spark conversion) -> ThreadScope
\end{itemize}



\section{semi-explicit parallelism (2)}

Combinator \codeline{pseq :: a -> b -> b} \codeline{e$_1$ `pseq` e$_2$}
evaluates \codeline{e$_1$} and then return \codeline{e$_2$} (sequentially on the same thread).

Example: Parallel evaluation of arguments in application\\
\begin{codebox}[haskell]
f x y = 
    x `par` (y `pseq` (f x y))
\end{codebox}



\section{Lazyness}

% \codefile{haskell}{caption={parFibPies.hs}, label={parFibPies.hs}}{../material/parFibPies.hs}

Lazy reductions of expressions to values:

\begin{enumerate}
    \item Subexpression are turned into \textbf{thunks} (values to be evaluated).
    \item A thunk is evaluated to WHBF (see below) \underline{only if it this demanded.}
    \item A thunk is never evaluated twice (overwritten with its value).
\end{enumerate}

\subsection{Weak Head Normal Form}

Expression \codeline{e} is in WHNF if it is
\begin{itemize}
    \item an atomic value
    \item a value constructor, possibly applied to arguments (e.g. \codeline{Just \_}, \codeline{\_:\_}, \codeline{(\_,\_)})
    \item a function (applied to too few arguments) (e.g. \codeline{\_ + \_}, \codeline{\textbackslash x -> \_})
\end{itemize}


\textit{Note:} The left argument of \codeline{pseq}/\codeline{par} is reduced to WHNF only!

% \codefile{haskell}{caption={parWHNF.hs}, label={parWHNF.hs}}{../material/parWHNF.hs}

% \begin{codebox}[haskell]
% -- really evaluate a list
% forceList :: [a] -> ()
% forceList []     = ()
% forceList (x:xs) = x `pseq` forceList xs
% \end{codebox}


% \codefile{haskell}{caption={NFData.hs}, label={NFData.hs}}{../material/NFData.hs}
\codefile{haskell}{caption={parFibPie-force.hs}, label={parFibPie-force.hs}}{../material/parFibPie-force.hs_ASCII}

\codefile{haskell}{caption={DeepSeq.hs}, label={DeepSeq.hs}}{../material/DeepSeq.hs}


\subsection{NFData}

\begin{itemize}
    \item Class \codeline{NFData a}: types \codeline{a} that support explicit evaluation to normal form (NF) via \codeline{rnf}
    \item Use \codeline{e$_1$ `deepseq` e$_2$} or \codeline{force e$_1$} to evaluate \codeline{e$_1$} to NF.
\end{itemize}


% \codefile{haskell}{caption={SuperParIndices.hs}, label={SuperParIndices.hs}}{../material/SuperParIndices.hs}

Creating sparks for too many / too small granules of computation is inefficient!
We need to control \textbf{granularity} of parallelism.

\codefile{haskell}{caption={ParIndices.hs}, label={ParIndices.hs}}{../material/ParIndices.hs}

\codefile{haskell}{caption={indices.hs}, label={indices.hs}}{../material/indices.hs}



\subsection{Monoid Reduction}

\begin{codebox}[haskell]
    f x@$_1$@ @$\oplus$@ (f x@$_2$@ @$\oplus$@ (f x@$_3$@ @$\oplus$@ f x@$_4$@))
@$\equiv$@ (f x@$_1$@ @$\oplus$@ f x@$_2$@) @$\oplus$@ (f x@$_3$@ @$\oplus$@ f x@$_4$@)
\end{codebox}

This structure is perfect for dividing a computation into thunks.

Monoids are
\begin{itemize}
    \item \codeline{((+), 0)}
    \item \codeline{((++), [])}
    \item \codeline{((\&\&), True)}
    \item \codeline{((||), False)}
    \item \dots
\end{itemize}


\codefile{haskell}{caption={ParFoldMap.hs}, label={ParFoldMap.hs}}{../material/ParFoldMap.hs}


\codefile{haskell}{caption={Wookie.txt}, label={Wookie.txt}}{../material/Wookie.txt_ASCII}

\codefile{haskell}{caption={WordState.hs}, label={WordState.hs}}{../material/WordState.hs_ASCII}

\codefile{haskell}{caption={allWords.hs}, label={allWords.hs}}{../material/allWords.hs}





% chapter parallelism (end)

