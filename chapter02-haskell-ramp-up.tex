%!TEX root = fp.tex

% Author: Philipp Moers <soziflip funny character gmail dot com>

\section{Haskell Ramp-Up} % (fold)
\label{cha:haskell_ramp_up}

(Read $\equiv$ as ''denotes the same value as'')

\begin{itemize}
    \item Apply f to value e: \codeline{f e} (juxtaposition, ''apply'', binary operator \textvisiblespace, Haskell speak: infixL 10 \textvisiblespace)
    \item \textvisiblespace\ has max precedence (10): \codeline{f |$e_1$| + |$e_2$| |$\equiv$| (f |$e_1$|) + |$e_2$|}
    \item \textvisiblespace\ associates to the left: \codeline{g f e |$\equiv$| (g f) e --(g f) is a function)} 
    \item Function composition:
    \begin{itemize}
        \item \codeline{(g . f) e |$\equiv$| g (f e) --(. is something like mathematical |$\circ$| ''after'')}
        \item Alternative ''apply''-operator \codeline{|\$|} (lowest precedence, associates to the right, infixR 0 \$):\\
            \codeline{g |\$| f |\$| e |$\equiv$| g |\$| (f |\$| e) |$\equiv$| g (f e)}
        \item Prefix application of binary infix operator $\otimes$: \codeline{|$(\otimes)$| |$e_1$| |$e_2$| |$\equiv$| |$e_1$| |$\otimes$| |$e_2$|} 
        \item Infix application of binary function f: \codeline{|$e_1$| `f` |$e_2$| |$\equiv$| f |$e_1$| |$e_2$|}:
        \begin{itemize}
            \item \codeline{1 `elem` [1,2,3] --|($1 \in \{1,2,3\}$)|}
            \item \codeline{n `mod` m}
            \item \dots
        \end{itemize}
        \item User defined operators, built from symbols \\ ! \# \$ \% \& * + / < = > ? \@ \textbackslash \string^ \textbar $\sim$:.
    \end{itemize}
\end{itemize}

\subsection{Function Application}
Any series of identifiers is a function call or, as we often call it, a function application.\\
\codeline{a b c d}\\
This is an application of a function a to three arguments b, c and d.\\
You may parenthesize function application if you need to.\\
\codeline{f a b |$\equiv$| (f a b) |$\not\equiv$| f (a, b)}\\
The last one is valid Haskell, but f is a function that takes a pair, \codeline{(a, b)} as an argument.
