%!TEX root = fp.tex

\section{Functors}
Type class Functor embodies the application of a function to the elements (or: inside) of a structure, which leaving structure (or: outside) alone.

\subsection{Examples}
\begin{Haskell}
map :: (a -> b) -> [a] -> [b]
mapTree :: (a -> b) -> Tree a -> Tree b

--Note: f is a type constructor that receives exactly one argument
--(Functor is also called a constructor class).
class Functor f where
   fmap :: (a -> b) -> f a -> f b
   
--Examples:
instance Functor [] where
   fmap = map
   
instance Functor Tree where
   fmap = mapTree
   
instance Functor Maybe where
   fmap f (Just x) = Just (f x)
   fmap f Nothing  = Nothing
\end{Haskell}

\begin{minipage}{\textwidth}
Type constructors can be partially applied. Uses prefix notation:\\
\begin{Haskell}
a -> b @$\equiv$@ (->) a b
(a, b) @$\equiv$@ (,) a b
[a] @$\equiv$@ [] a

--Examples (defines type constructors):
type Flagged = (,) Bool
type Indexed (->) Int
--MayFail e a: computation may yield value a or fail with error e
type MayFail e = Either e

instance Functor (Either e) where
   fmap f (Left e)  = Left e
   fmap f (Right x) = Right (f x)
   
instance Functor Flagged where
--fmap :: (a -> b) -> (Bool, a) -> (Bool, b)
   fmap f (b, x) = (b, f x)
   
instance Functor Indexed where
   fmap f g = f . g

fmap id @$\equiv$@ id
fmap (f . g) = fmap f . fmap g
\end{Haskell}
\end{minipage}

%\Haskellcode{../material/round.hs}