%!TEX root = fp.tex

\section{Functors}
Type class Functor embodies the application of a function to the elements (or: inside) of a structure, which leaving structure (or: outside) alone.

\subsection{Examples}
\begin{Haskell}
map :: (a -> b) -> [a] -> [b]
mapTree :: (a -> b) -> Tree a -> Tree b

--Note: f is a type constructor that receives exactly one argument
--(Functor is also called a constructor class).
class Functor f where
   fmap :: (a -> b) -> f a -> f b
   
--Examples:
instance Functor [] where
   fmap = map
   
instance Functor Tree where
   fmap = mapTree
   
instance Functor Maybe where
   fmap f (Just x) = Just (f x)
   fmap f Nothing  = Nothing
\end{Haskell}

\begin{minipage}{\textwidth}
Type constructors can be partially applied. Uses prefix notation:\\
\begin{Haskell}
a -> b @$\equiv$@ (->) a b
(a, b) @$\equiv$@ (,) a b
[a] @$\equiv$@ [] a

--Examples (defines type constructors):
type Flagged = (,) Bool
type Indexed (->) Int
--MayFail e a: computation may yield value a or fail with error e
type MayFail e = Either e

instance Functor (Either e) where
   fmap f (Left e)  = Left e
   fmap f (Right x) = Right (f x)
   
instance Functor Flagged where
--fmap :: (a -> b) -> (Bool, a) -> (Bool, b)
   fmap f (b, x) = (b, f x)
   
instance Functor Indexed where
   fmap f g = f . g

--Functor Laws
fmap id @$\equiv$@ id
fmap (f . g) = fmap f . fmap g
\end{Haskell}
\end{minipage}
%\Haskellcode{../material/round.hs}

\subsection{Kinds - Types for Types}

\begin{center}\begin{tabular}{|c|c|c|}\hline
\rowcolor{grau} kind           & describes...                   & example\\\hline
\codeline{*}                 & types                             & \codeline{Int, Bool, (Bool, Int), [Char], ...} \\\hline
\codeline{* -> *}         & unary type constructor  & \codeline{Maybe,  []}\\\hline
\codeline{* -> * -> *} & binary type constructor & \codeline{Either, (,), (->)}\\\hline
\end{tabular}\end{center}


\section{Applicative}
\begin{Haskell}
--Compare:
(@$\$$@) :: (a -> b) -> a -> b
fmap :: Functor f => (a -> b) -> f a -> f b --<@$\$$@>
(<*>) :: Applicative f => f (a -> b) -> f a -> f b
\end{Haskell}

Read \codeline{<*>} as (ap)ply, ''tiefighter''.

\begin{Haskell}
class Functor f => Applicative f where
   pure :: a -> f a
   (<*>) :: f (a -> b) -> f a -> f b
   
--Make any Applicative f a Functor:
fmap g x = pure g <*> x
\end{Haskell}

Applicative embodies
\begin{enumerate}
\item function \underline{application} on the level of (contained) values.
\item \underline{combination} on the level of structures.
\end{enumerate}

\subsection{Applicative instances}
\begin{Haskell}
instance Applicative Maybe where
   pure x = Just x
   Just f <*> Just x = Just (f x)
   _ <*> _ = Nothing
   
instance Monoid c => Applicative ((,) c) where
   pure x = (mempty, x)
   (@$c_1$@, f) <*> (@$c_2$@, x) = (@$c_1$@ `mappend`@$c_2$@, f x)
   
instance Applicative [] where
   pure x = [x]
   fs <*> xs = [f x | f <- fs, x <- xs]
\end{Haskell}

\section{Interlude: Monoid}

Type class Monoid a represents combinable values of type a:

\begin{Haskell}
class Monoid a where
   mempty :: a -- empty, neutral element for mappend
   mappend :: a -> a -> a -- combination
   mconcat :: [a] -> a
\end{Haskell}

Examples: $(0, +), (1, \cdot), (\text{true}, \land), (\text{false}, \lor), (\{\}, \cup), ([], (++))$

\subsection{Monoid Laws}
\begin{Haskell}
mempty `mappend`xs @$\equiv$@ xs
xs `mappend`(ya `mappend`zs) @$\equiv$@ (xs `mappend`ys) `mappend`zs``
mconcat xs @$\equiv$@ foldr mempty mappend xs
\end{Haskell}