%!TEX root = fp.tex

% Author: Philipp Moers <soziflip funny character gmail dot com>

\section{Values and Types} % (fold)
\label{cha:values_and_types}

Any Haskell expression e has a type t (\codeline{e :: t}) that is determined at compile time.
The \textbf{type assigmnent ::} is either given explicitly or inferred by the compiler.

\section{Base Types}
\begin{center}
\begin{tabular}{|c|c|c|}\hline
\rowcolor{grau}     
Type					& Description                                   	& values                                \\\hline
Int						  & fixed-prec. integer                            & 0, 1, (-42)                           \\\hline
Integer         	   & arbitrary prec. integer                        & 10\textasciicircum 100                \\\hline
Float, Double      & single/double floating point (IEEE)    & 0.1, 1e02                             \\\hline
Char            		& Unicode character                             & ``x'', ``\textbackslash t'',  ``$\triangle$'', ``\textbackslash 8710''\\\hline
Bool            		& Boolean                                       	  & True, False                           \\\hline
()              		   & Unit                                          			& ()                                    \\\hline
\end{tabular}
\end{center}


\section{Type Constructors}

\begin{itemize}
    \item Build new types from existing types
    \item Let a, b \dots denote arbitrary types (\textbf{type variables})
\end{itemize}
\begin{center}
\begin{tabular}{|c|c|c|}\hline
\rowcolor{grau}     Type            & Description                                   & values                        \\\hline
\codeline{(a, b)}          & pairs of values of type a, b                  & \codeline{(1, True) :: (Int, Bool)}      \\\hline
\codeline{(a|$_1$|, a|$_2$|, |\dots| a|$_n$|)} & n-tuples                          &                               \\\hline
\codeline{[a]}             & list of values of type a                      & \codeline{[True, False] :: [Bool], []::[a]}          \\\hline
\codeline{Maybe a}         & optional value of type a                      & \multirow{2}{3.7cm}{\codeline{Just 42 :: Maybe Int} 
                                                                                         \codeline{Nothing :: Maybe a}}    \\
                    &                                               &                               \\\hline
\codeline{Either a b}      & choice                                        & \multirow{2}{5cm}{\codeline{Left 'x' :: Either Char b}
                                                                                     \codeline{Right pi :: Either a Double}}   \\
                    &                                               &                               \\\hline
\codeline{IO a}            & \multirow{2}{4.2cm}{I/O actions that 
                                            return a value of type a}   & \codeline{print 42 :: IO ()}             \\
                    &                                               &                               \\\hline
\codeline{a -> b} & functions from a to b                       & \codeline{isLetter :: Char -> Bool}      \\\hline
\end{tabular}
\end{center}

\section{Currying}

\begin{itemize}
  \item \textit{Recall}: \codeline{e|$_1$| ++ e|$_2$| |$\equiv$| (++) e|$_1$| e|$_2$|}
  \item \codeline{(++) e|$_1$| e|$_2$| |$\equiv$| ((++) e|$_1$|) e|$_2$|}
  \item Function application happens one argument at a time. \\ (\textbf{Currying}, Haskell B. Curry)
  \item Type of n-ary function is \\ \codeline{a|$_1$| -> a|$_2$| -> |\dots| a|$_n$| -> b}
  \item Type fun -> associates to the right, read above type as \\ \codeline{a|$_1$| -> (a|$_2$| -> (\dots (|$a_n$| -> |$b$|)))}
  \item Enables \textbf{Partial Application}
\end{itemize}


\section{Defining Values (and thus functions)}

\begin{itemize}
  \item \codeline{=} binds names to values. Names must not start with A-Z (Haskell style: camelCase)
  \item Define constant (0-ary function) c. Value of c is value of expression e. \\ \codeline{c = e}
  \item Define n-ary function f with arguments x$_i$. f may occur in e. \\ \codeline{f |$x_1$| |$x_2$| \dots |$x_n$| = e}
  \item A Haskell program is a set of bindings.
  \item Good style: give type assigmnents for top-level (global) bindings:\\
  \begin{Haskell}
f :: a@$_1$@ -> a@$_2$@ -> b
f x@$_1$@ x@$_2$@ = e
  \end{Haskell}
\end{itemize}

\subsection{Guards}

Guards are conditional expressions (something like ``switch'' in Java).
They are a lot more readable and more powerful than \codeline{if |\dots| then |\dots| else |\dots|}.

Guards are introduced by \codeline{|}:\\
\begin{Haskell}
f x@$_1$@ x@$_2$@ @\dots@ x@$_n$@
  | q@$_1$@     = e@$_1$@
  | q@$_2$@     = e@$_2$@
  @\dots@
  | q@$_m$@     = e@$_m$@
[ | otherwise   = e@$_{m+1}$@ ]
\end{Haskell}

Guards (q$_i$) are expressions of type Bool, evaluated top to bottom.

\Haskellcode{../material/factorial.hs}

\subsection{Local Definitions}

\begin{enumerate}
  \item \textbf{Where bindings}: local definitions visible in the entire rhs of a definition.\\
  \begin{Haskell}
f@$_1$@ x@$_1$@ x@$_2$@ @\dots@ x@$_n$@ | q@$_1$@ = e@$_1$@
                    | q@$_2$@ = e@$_2$@ 
                    @\dots@
                    | q@$_m$@ = e@$_m$@ 
	where 
		g@$_1$@ = @\dots@
		g@$_2$@ = @\dots@
		@\dots@
		g@$_o$@
  \end{Haskell}

  \Haskellcode{../material/power.hs}

  \item \textbf{Let expressions}: local definitions visible inside one expression.\\
  \begin{Haskell}
let g@$_1$@ = @\dots@
    g@$_2$@ = @\dots@
    @\dots@
    g@$_o$@
in e
  \end{Haskell}
\end{enumerate}

\subsection{Lists}

\begin{itemize}
  \item Recursive definitions:
  \begin{enumerate}
      \item \codeline{[]} is a list (nil), type \codeline{[] :: [a]}
      \item \codeline{x:xs} is a list, if \codeline{x :: a, xs :: [a]}\\
      (x is head, xs is tail)
  \end{enumerate}
  \item Notation: \codeline{3:(2:(1:[]))} $\equiv$ \codeline{3:2:1:[]} $\equiv$ \codeline{[3,2,1]} $\equiv$ \codeline{3:[2,1]}
  \item Law: $\forall$ xs :: [a] :   \hspace{1cm} (xs $\neq$ []) \\
      \codeline{head xs : tail xs} == xs
\end{itemize}

\subsection{Pattern Matching}

\begin{itemize}
  \item \textit{The} idiomatic Haskell way to define a function by cases:\\
  \begin{Haskell}
f :: a@$_1$@ -> @\dots@ a@$_n$@ -> b
f p@$_11$@ @\dots@ p@$_1k$@ = e@$_1$@
f p@$_21$@ @\dots@ p@$_2k$@ = e@$_2$@
@\dots@
f p@$_n1$@ @\dots@ p@$_nk$@ = e@$_k$@
  \end{Haskell}

\end{itemize}

\vspace{9pt}\begin{center}\begin{tabular}{|c|c|c|}\hline
\rowcolor{grau}   
Pattern         & Matches If                & Bindings in e$_r$     \\\hline
  constant c      & x$_i$ == c                  &                     \\\hline
  variable v      & always                    & v $\equiv$ x$_i$      \\\hline
  wildcard \_      & always                    &                       \\\hline
  tuple (p$_1$, \dots p$_m$)  & components of x$_i$ match patterns p    & \\\hline
  []              & x$_i$ == []                 &                     \\\hline
  (p$_1$ : p$_2$)     & head x$_i$ matches p$_1$, tail x$_i$ matches p$_2$    & \\\hline
\end{tabular}\end{center}\vspace{9pt}

\Haskellcode{../material/tally.hs}
\Haskellcode{../material/take.hs}
\Haskellcode{../material/mergesort.hs}


\section{Algebraic Data Types}

(also known as \textbf{Sum-of-Product-Types})

\begin{itemize}
  \item \textit{Recall}: \codeline{[]} and \codeline{(:)} are the \textbf{values constructors} for \textbf{type constructor} [a]. 
  \item Can define entirely new type T and its constructors K$_i$:\\
        \begin{Haskell}
data T a@$_1$@ a@$_2$@ @\dots@ a@$_n$@ = K@$_1$@ b@$_{11}$@ @\dots@ b@$_{1_{n_1}}$@
                     K@$_2$@ b@$_{21}$@ @\dots@ b@$_{2_{n_2}}$@
                     @\dots@
                     K@$_r$@ b@$_{r1}$@ @\dots@ b@$_{r_{n_r}}$@
        \end{Haskell}
        
        b$_{ij}$ types mentioning the type vars a$_1$ \dots a$_n$

  \item Defines type constructor T and r value constructors:\\
        \codeline{K|$_i$| :: b|$_{i_1}$| -> b|$_{i_2}$| -> |\dots| b|$_{i_n}$| -> T a|$_1$| |\dots| a|$_n$|}
  \item Compare \codeline{[] :: [a]} and \codeline{(:) :: a -> [a] -> [a]}
  \item \textbf{Sum Type} (n=0, n$_i$ = 0) \\
%        \Haskellcode{../material/weekday.hs}
\Haskellcode{../material/deriving.hs}

  \item Add \codeline{deriving (c, c, |\dots| c)} to data declaration to define canonical operations: 
\begin{center}\begin{tabular}{|c|c|}\hline
  \rowcolor{grau} c       & operations                          \\\hline
                  Eq      & equality (==, /=)                   \\\hline
                  Show    & printing (show)                     \\\hline
                  Ord     & ordering ($<$, $<=$, max)               \\\hline
                  Enum    & enumeration                         \\\hline
                  Bounded & minBound, maxBound                  \\\hline
  \end{tabular}\end{center}
  % \codefile{haskell}{caption={deriving.hs}, label=deriving.hs}{../material/deriving.hs}
  \item \textbf{Product Types} (r=1)\\
  \Haskellcode{../material/sequence.hs}
  \item \textbf{Sum-of-Product-Types}\\
\begin{Haskell}
data Maybe a = Just a | Nothing
data Either a b = Left a | Right b
data List a = Nil | Cons a (List a)
\end{Haskell}

    \Haskellcode{../material/cons.hs}

\begin{Haskell}
-- Use the isomorphism between [a] and List a
-- to save work when defining functions over List a:
--
--                 fromList
--       List a -------------→ [a]
--         |                    |
--       g |                    | f
--         ↓                    ↓
--       List b ←------------ [b]
\end{Haskell}

\Haskellcode{../material/eval.hs}
        
\end{itemize}
