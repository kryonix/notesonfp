% Author: Philipp Moers <soziflip funny character gmail dot com>

% \documentclass[12pt, oneside, a4paper, numbers=enddot, abstracton]{scrreprt}
\documentclass[12pt,a4paper]{article}

\usepackage[utf8x]{inputenc}
\usepackage{lscape}
\usepackage[ngerman, english]{babel}
% \usepackage{fancyhdr}
\usepackage{amsmath}
\usepackage{amsthm}
\usepackage{amsfonts}
\usepackage{amssymb}
\usepackage{mathtools}
\usepackage{array}
\usepackage{paralist}
\usepackage{tabularx}
\usepackage{listings}
\usepackage{multicol}
\usepackage{mdframed}
\usepackage{url}
\usepackage[lined,boxed,commentsnumbered]{algorithm2e}
\usepackage{xcolor}
\usepackage{minted}
\usepackage{datetime}
\usepackage{colortbl}
\usepackage{multirow}

\definecolor{grau}{rgb}{0.75,0.75,0.75}

\definecolor{mintedbackground}{rgb}{0.95,0.95,0.95}
\newmintedfile[Haskellcode]{Haskell}{
bgcolor=mintedbackground,
fontfamily=tt,
linenos=true,
numberblanklines=true,
numbersep=12pt,
numbersep=5pt,
gobble=0,
frame=leftline,
framerule=0.4pt,
framesep=2mm,
funcnamehighlighting=true,
tabsize=4,
obeytabs=false,
escapeinside=@@,
mathescape=true
samepage=false, %with this setting you can force the list to appear on the same page
showspaces=false,
showtabs =false,
texcl=false,
}

\newminted[Haskell]{Haskell}{
bgcolor=mintedbackground,
fontfamily=tt,
linenos=true,
numberblanklines=true,
numbersep=12pt,
numbersep=5pt,
gobble=0,
frame=leftline,
framerule=0.4pt,
framesep=2mm,
funcnamehighlighting=true,
tabsize=4,
obeytabs=false,
escapeinside=@@,
mathescape=true
samepage=false, %with this setting you can force the list to appear on the same page
showspaces=false,
showtabs =false,
texcl=false,
}

%\usepackage{algpseudocode}

\usepackage{textcomp}
%\usepackage{ulsy}
\usepackage{array}
%Bäume, Automaten:
%\usepackage{qtree}
\usepackage{pgf}
\usepackage{tikz}
\usetikzlibrary{arrows,automata}

\usepackage{calc}
\usepackage{wasysym}

\newlength\celldim \newlength\fontheight \newlength\extraheight
\newcounter{sqcolumns}

\newcolumntype{S}{ @{}
  >{\centering \rule[-0.5\extraheight]{0pt}{\fontheight + \extraheight}}
  p{\celldim} @{} }

\newcolumntype{Z}{ @{} >{\centering} p{\celldim} @{} }

\newenvironment{squarecells}[1]
  {\setlength\celldim{3em}%
   \settoheight\fontheight{A}%
   \setlength\extraheight{\celldim - \fontheight}%
   \setcounter{sqcolumns}{#1 - 1}%
   \begin{tabular}{|S|*{\value{sqcolumns}}{Z|}}\hline}
% squarecells tabular goes here
  {\end{tabular}}

%\newcommand\nl{\tabularnewline\hline}

\author{Denis Hirn}
\title{FP Additional}

\usepackage{fancyhdr}
\pagestyle{fancy}

\renewcommand{\headrulewidth}{1pt}
\renewcommand{\footrulewidth}{0.4pt}

\renewcommand{\i}{\mathtt{i}}
\renewcommand{\b}{\mathtt{b}}
\newcommand{\integer}{\ensuremath{\mathtt{integer}}}

\addtolength{\textwidth}{80px}
\addtolength{\hoffset}{-40px}
\addtolength{\topmargin}{-30px}
\addtolength{\textheight}{30px}

\setlength{\headheight}{41.6px}

\fancyheadoffset{20pt}

\newcommand{\bbl}{\textlbrackdbl{}}
\newcommand{\bbr}{\textrbrackdbl{}}
\newcommand{\Pot}{\ensuremath{\mathcal{P}}}

\newcommand{\qq}[1]{\glqq #1 \grqq}
\newcommand{\N}{\ensuremath{\mathbb{N}}}
\newcommand{\Z}{\ensuremath{\mathbb{Z}}}
\newcommand{\R}{\ensuremath{\mathbb{R}}}
\newcommand{\Q}{\ensuremath{\mathbb{Q}}}
\newcommand{\0}{\ensuremath{\mathcal{O}}}

\newcommand{\abs}[1]{\ensuremath{ | #1 |}}
\newcommand{\dol}{\ensuremath{\$ {}}}
\newcounter{iabc}
\newenvironment{abc}{\begin{list}{\alph{iabc})}{\usecounter{iabc}}}{\end{list}}


\begin{document}
% \selectlanguage{english}

\begin{titlepage}
    \begin{center}
        \includegraphics[width=0.7\textwidth]{logo-uni-tuebingen}\\[1cm]

        % Title
        \newcommand{\HRule}{\rule{\linewidth}{0.5mm}} \HRule \\[0.4cm]
        { \huge \bfseries Functional Programing}\\[0.4cm]

        \textsc{\Large Sommersemester 2014, Wintersemester 2015}\\[0.5cm]
        \HRule \\[1.5cm]

        \begin{minipage}{0.4\textwidth}
            \begin{flushleft}
                \large \emph{Author:}\\ Philipp Moers,\\Update by: Denis Hirn
            \end{flushleft}
        \end{minipage}
            \hfill
        \begin{minipage}{0.4\textwidth}
            \begin{flushright}
                \large \emph{Dozent:} \\ \scshape{Torsten Grust, Alexander Ulrich}
            \end{flushright}
        \end{minipage}

        \vfill
        {Last updated: \today, \currenttime}
    \end{center}
\end{titlepage}

\begin{abstract}
    This is just the product of me taking notes on the lecture. Nothing official. If you find mistakes or have got any questions, please feel free to contact me. Cheers!
\end{abstract}


\tableofcontents


\newpage

\vspace*{\fill}
»\textit{A programming language is a medium for expressing ideas (not to get a computer to perform operations) and only incidentally for machines to execute.}«\\
\begin{flushright}
    Harold Abelson and Gerald Jay Sussman
\end{flushright}
\vspace*{\fill}

\newpage


\section*{Links}

Site 2014: \url{http://db.inf.uni-tuebingen.de/teaching/FunctionalProgrammingSS2014.html}\\
Site 2015: \url{http://db.inf.uni-tuebingen.de/teaching/FunctionalProgrammingWS2015-2016.html}
Ilias: \url{http://goo.gl/rlqbkK}

\section*{Literature}

\begin{itemize}
	\item Bird:\\Thinking Functionally with Haskell, Cambridge University Press 2014\\ \url{http://www.cs.ox.ac.uk/publications/books/functional/}
	\item Keller, Chakravarthy: ''Learning Haskell'', online course in development\\ \url{http://learn.hfm.io/}
    \item Lipovača: \\ Learn You a Haskell for Great Good \\ No Starch Press 2011, \\ \url{http://learnyouahaskell.com}
    \item O'Sullivan, Steward, Goerzen: \\ Real World Haskell \\ O'Reilly 2010 \\ \url{http://book.realworldhaskell.org}
    \item Haskell 2010 Report, \\ \url{http://www.haskell.org/onlinereport/haskell2010}
\end{itemize}


% chapters:
\pagebreak
\input{chapter01-introduction.tex}

\pagebreak
\newcommand{\codeline}[1]{\mintinline[escapeinside=||,mathescape=true]{Haskell}{#1}}
\input{chapter02-haskell-ramp-up.tex}

\pagebreak
%!TEX root = fp.tex

% Author: Philipp Moers <soziflip funny character gmail dot com>

\section{Values and Types} % (fold)
\label{cha:values_and_types}

Any Haskell expression e has a type t (\codeline{e :: t}) that is determined at compile time.
The \textbf{type assigmnent ::} is either given explicitly or inferred by the compiler.

\subsection{Base Types}
\begin{center}
\begin{tabular}{|c|c|c|}\hline
\rowcolor{grau}     
Type					& Description                                   	& values                                \\\hline
Int						  & fixed-prec. integer                            & 0, 1, (-42)                           \\\hline
Integer         	   & arbitrary prec. integer                        & 10\textasciicircum 100                \\\hline
Float, Double      & single/double floating point (IEEE)    & 0.1, 1e02                             \\\hline
Char            		& Unicode character                             & ``x'', ``\textbackslash t'',  ``$\triangle$'', ``\textbackslash 8710''\\\hline
Bool            		& Boolean                                       	  & True, False                           \\\hline
()              		   & Unit                                          			& ()                                    \\\hline
\end{tabular}
\end{center}


\subsection{Type Constructors}

\begin{itemize}
    \item Build new types from existing types
    \item Let a, b \dots denote arbitrary types (\textbf{type variables})
\end{itemize}
\begin{center}
\begin{tabular}{|c|c|c|}\hline
\rowcolor{grau}     Type            & Description                                   & values                        \\\hline
\codeline{(a, b)}          & pairs of values of type a, b                  & \codeline{(1, True) :: (Int, Bool)}      \\\hline
\codeline{(a|$_1$|, a|$_2$|, |\dots| a|$_n$|)} & n-tuples                          &                               \\\hline
\codeline{[a]}             & list of values of type a                      & \codeline{[True, False] :: [Bool], []::[a]}          \\\hline
\codeline{Maybe a}         & optional value of type a                      & \multirow{2}{3.7cm}{\codeline{Just 42 :: Maybe Int} 
                                                                                         \codeline{Nothing :: Maybe a}}    \\
                    &                                               &                               \\\hline
\codeline{Either a b}      & choice                                        & \multirow{2}{5cm}{\codeline{Left 'x' :: Either Char b}
                                                                                     \codeline{Right pi :: Either a Double}}   \\
                    &                                               &                               \\\hline
\codeline{IO a}            & \multirow{2}{4.2cm}{I/O actions that 
                                            return a value of type a}   & \codeline{print 42 :: IO ()}             \\
                    &                                               &                               \\\hline
\codeline{a -> b} & functions from a to b                       & \codeline{isLetter :: Char -> Bool}      \\\hline
\end{tabular}
\end{center}

\subsection{Currying}

\begin{multicols}{2}
\begin{itemize}
  \item \textit{Recall}: \codeline{e|$_1$| ++ e|$_2$| |$\equiv$| (++) e|$_1$| e|$_2$|}
  \item \codeline{(++) e|$_1$| e|$_2$| |$\equiv$| ((++) e|$_1$|) e|$_2$|}
  \item Function application happens one argument at a time. \\ (\textbf{Currying}, Haskell B. Curry)
  \item Type of n-ary function is \\ \codeline{a|$_1$| -> a|$_2$| -> |\dots| a|$_n$| -> b}
  \item Type fun \codeline{->} associates to the right, read above type as \\ \codeline{a|$_1$| -> (a|$_2$| -> (\dots (|$a_n$| -> |$b$|)))}
  \item Enables \textbf{Partial Application}
\end{itemize}
\end{multicols}


\subsection{Defining Values (and thus functions)}

\begin{itemize}
  \item \codeline{=} binds names to values. Names must not start with A-Z (Haskell style: camelCase)
  \item Define constant (0-ary function) c. Value of c is value of expression e. \\ \codeline{c = e}
  \item Define n-ary function f with arguments x$_i$. f may occur in e. \\ \codeline{f |$x_1$| |$x_2$| |\dots| |$x_n$| = e}
  \item A Haskell program is a set of bindings.
  \item Good style: give type assigmnents for top-level (global) bindings:\\
  \begin{Haskell}
f :: a@$_1$@ -> a@$_2$@ -> b
f x@$_1$@ x@$_2$@ = e
  \end{Haskell}
\end{itemize}

\subsubsection{Guards}

Guards are conditional expressions (something like ``switch'' in Java).
They are a lot more readable and more powerful than \codeline{if |\dots| then |\dots| else |\dots|}.

Guards are introduced by \codeline{|}:\\
\begin{Haskell}
f x@$_1$@ x@$_2$@ @\dots@ x@$_n$@
  | q@$_1$@     = e@$_1$@
  | q@$_2$@     = e@$_2$@
  @\dots@
  | q@$_m$@     = e@$_m$@
[ | otherwise   = e@$_{m+1}$@ ]
\end{Haskell}

Guards (q$_i$) are expressions of type Bool, evaluated top to bottom.

\Haskellcode{../material/factorial.hs}

\subsubsection{Local Definitions}

\begin{enumerate}
  \item \textbf{Where bindings}: local definitions visible in the entire rhs of a definition.\\
  \begin{Haskell}
f@$_1$@ x@$_1$@ x@$_2$@ @\dots@ x@$_n$@ | q@$_1$@ = e@$_1$@
                    | q@$_2$@ = e@$_2$@ 
                    @\dots@
                    | q@$_m$@ = e@$_m$@ 
	where 
		g@$_1$@ = @\dots@
		g@$_2$@ = @\dots@
		@\dots@
		g@$_o$@
  \end{Haskell}

  \Haskellcode{../material/power.hs}

  \item \textbf{Let expressions}: local definitions visible inside one expression.\\
  \begin{Haskell}
let g@$_1$@ = @\dots@
    g@$_2$@ = @\dots@
    @\dots@
    g@$_o$@
in e
  \end{Haskell}
\end{enumerate}

\subsubsection{Lists}

\begin{itemize}
  \item Recursive definitions:
  \begin{enumerate}
      \item \codeline{[]} is a list (nil), type \codeline{[] :: [a]}
      \item \codeline{x:xs} is a list, if \codeline{x :: a, xs :: [a]}\\
      (x is head, xs is tail)
  \end{enumerate}
  \item Notation: \codeline{3:(2:(1:[]))} $\equiv$ \codeline{3:2:1:[]} $\equiv$ \codeline{[3,2,1]} $\equiv$ \codeline{3:[2,1]}
  \item Law: $\forall$ xs :: [a] :   \hspace{1cm} (xs $\neq$ []) \\
      \codeline{head xs : tail xs} == xs
\end{itemize}

\subsubsection{Pattern Matching}

\begin{itemize}
  \item \textit{The} idiomatic Haskell way to define a function by cases:\\
  \begin{Haskell}
f :: a@$_1$@ -> @\dots@ a@$_n$@ -> b
f p@$_11$@ @\dots@ p@$_1k$@ = e@$_1$@
f p@$_21$@ @\dots@ p@$_2k$@ = e@$_2$@
@\dots@
f p@$_n1$@ @\dots@ p@$_nk$@ = e@$_k$@
  \end{Haskell}

\end{itemize}
\begin{center}
\begin{tabular}{|c|c|c|}\hline
\rowcolor{grau}   
Pattern         & Matches If                & Bindings in e$_r$     \\\hline
  constant c      & x$_i$ == c                  &                     \\\hline
  variable v      & always                    & v $\equiv$ x$_i$      \\\hline
  wildcard \_      & always                    &                       \\\hline
  tuple (p$_1$, \dots p$_m$)  & components of x$_i$ match patterns p    & \\\hline
  []              & x$_i$ == []                 &                     \\\hline
  (p$_1$ : p$_2$)     & head x$_i$ matches p$_1$, tail x$_i$ matches p$_2$    & \\\hline
\end{tabular}
\end{center}

\Haskellcode{../material/tally.hs}
\Haskellcode{../material/take.hs}
\Haskellcode{../material/mergesort.hs}


\subsection{Algebraic Data Types}

(also known as \textbf{Sum-of-Product-Types})

\begin{itemize}
  \item \textit{Recall}: \codeline{[]} and \codeline{(:)} are the \textbf{values constructors} for \textbf{type constructor} [a]. 
  \item Can define entirely new type T and its constructors K$_i$:\\
        \begin{Haskell}
data T a@$_1$@ a@$_2$@ @\dots@ a@$_n$@ = K@$_1$@ b@$_{11}$@ @\dots@ b@$_{1_{n_1}}$@
                     K@$_2$@ b@$_{21}$@ @\dots@ b@$_{2_{n_2}}$@
                     @\dots@
                     K@$_r$@ b@$_{r1}$@ @\dots@ b@$_{r_{n_r}}$@
        \end{Haskell}
        
        b$_{ij}$ types mentioning the type vars a$_1$ \dots a$_n$

  \item Defines type constructor T and r value constructors:\\
        \codeline{K|$_i$| :: b|$_{i_1}$| -> b|$_{i_2}$| -> |\dots| b|$_{i_n}$| -> T a|$_1$| |\dots| a|$_n$|}
  \item Compare \codeline{[] :: [a]} and \codeline{(:) :: a -> [a] -> [a]}
  \item \textbf{Sum Type} (n=0, n$_i$ = 0) \\
%        \Haskellcode{../material/weekday.hs}
\Haskellcode{../material/deriving.hs}

  \item Add \codeline{deriving (c, c, |\dots| c)} to data declaration to define canonical operations: 
\begin{center}\begin{tabular}{|c|c|}\hline
  \rowcolor{grau} c       & operations                          \\\hline
                  Eq      & equality (==, /=)                   \\\hline
                  Show    & printing (show)                     \\\hline
                  Ord     & ordering ($<$, $<=$, max)               \\\hline
                  Enum    & enumeration                         \\\hline
                  Bounded & minBound, maxBound                  \\\hline
  \end{tabular}\end{center}
  % \codefile{haskell}{caption={deriving.hs}, label=deriving.hs}{../material/deriving.hs}
  \item \textbf{Product Types} (r=1)\\
  \Haskellcode{../material/sequence.hs}
  \item \textbf{Sum-of-Product-Types}\\
\begin{Haskell}
data Maybe a = Just a | Nothing
data Either a b = Left a | Right b
data List a = Nil | Cons a (List a)
\end{Haskell}

\Haskellcode{../material/cons.hs}

\begin{Haskell}
-- Use the isomorphism between [a] and List a
-- to save work when defining functions over List a:
--
--                 fromList
--       List a -------------→ [a]
--         |                    |
--       g |                    | f
--         ↓                    ↓
--       List b ←------------ [b]
\end{Haskell}

\Haskellcode{../material/eval.hs}
        
\end{itemize}


\pagebreak
%!TEX root = fp.tex

% Author: Philipp Moers <soziflip funny character gmail dot com>

\section{Type Classes} % (fold)
\label{cha:type_classes}

A \textbf{type class} C defines a family of type signatures (''methods'') which all \textbf{instances} of C must implement.

\begin{Haskell}
class C a where
    f@$_1$@ :: t@$_1$@
    @\dots@
    f@$_n$@ :: t@$_n$@
\end{Haskell}

The \codeline{t|$_i$|} \underline{must} mention a.\\
For any \codeline{f|$_i$|} the class may provide default implementations. \\
We have \codeline{f|$_i$| :: C a => t|$_i$|} \\ (read ''if a is instance C then f$_i$ has type t$_i$'').\\
\codeline{C a} is called \textbf{class constraint}.

\textit{Example}:

\begin{Haskell}
class Eq a where
    (==) :: a -> a -> Bool
    (/=) :: a -> a -> Bool
    x == y = not (x /= y)
    x /= y = not (x == y)
\end{Haskell}

(These are default implementations. To redefine one of them is sufficient.)

% \codefile{haskell}{caption={type-classes.hs}, label=type-classes.hs}{../material/type-classes.hs}


\subsection{Class Inheritance}

\begin{itemize}
    \item Defining \codeline{class (c|$_1$| a, c|$_2$| a, |\dots|) => C a where |\dots|} makes type class C a subclass of the C$_i$.
    \item \codeline{C a => t} implies \codeline{C|$_1$| a, C|$_2$| a |\dots|}.
\end{itemize}

% \vspace{9pt}
% \includegraphics[scale=1.0]{type-class-inheritance.png}
% \textcolor{myorange}{[Here is an image missing]}
% \vspace{9pt}


\subsection{Class Instances}

If type t implements the methods of class C, t becomes an \textbf{instance of} C:\\
\begin{Haskell}
instance C t where
    f@$_1$@ = <def of f@$_1$@>
    @\dots@
    f@$_n$@ = <def of f@$_n$@>
\end{Haskell}

(All defs of f$_i$ may be provided, minimal complete definition \underline{must} be provided.)
Class constraint C t is satisfied from now on.

\textit{Example}:\\
\begin{Haskell}
instance Eq Bool where
    x == y = x && y || (not x && not y)
\end{Haskell}

An instance definition for type constructor t may formulate class constraints for its argument types a, b, \dots:
\codeline{instance (C|$_1$| a, C|$_2$| a, |\dots|) => C t where}

\begin{mdframed}[linecolor=black, topline=false, bottomline=false,
  leftline=false, rightline=false, backgroundcolor=mintedbackground]
    \inputminted[fontfamily=tt]{Haskell}{../material/rock-paper-scissor-inst.hs}
    %\Haskellcode{../material/rock-paper-scissor-inst.hs}
\end{mdframed}

\subsubsection{Deriving Class Instances}
Automatically make user-defined data types (\codeline{data |\dots|}) instances of classes C$_i$ $\in$ \{ Eq, Ord, Enum, Bounded, Show, Read \}:

\begin{Haskell}
data T a@$_1$@ a@$_1$@ @\dots@ a@$_n$@ = @\dots@
                   | @\dots@
    deriving (C@$_1$@, C@$_2$@, @\dots@)
\end{Haskell}

\Haskellcode{../material/rock-paper-scissor-inst-deriving.hs_}

% chapter type_classes (end)


\pagebreak
%!TEX root = fp.tex

% Author: Philipp Moers <soziflip funny character gmail dot com>


\section{Domain-Specific Languages} % (fold)
\label{cha:domain_specific_languages}
a.k.a. \textbf{DSLs}

\begin{itemize}
    \item ''small'' languages designed to easily and directly express the concepts/idioms of a specific domain. \underline{Not} Turing-complete in general.
    \item Examples:
        \begin{center}\begin{tabular}{|c|c|}\hline
        \rowcolor{grau} Domain          & DSLs                                  \\\hline
                        OS automation   & shell scripts, OSX Automater          \\\hline
                        Typesetting     & \LaTeX                                \\\hline
                        Queries         & SQL                                   \\\hline
                        Game Scripting  & Unreal Script, Lua                    \\\hline
                        Parsing         & Yacc, Bison, ANTLR                    \\\hline
        \end{tabular}\end{center}
    \item Functional Languages make good hosts for \textbf{embedded DSLs}:
        \begin{itemize}
            \item algebraic data types (e.g. to model ASTs)
            \item higher-order functions (abstraction, control constructs)
            \item lightweight syntax (layout / whitespace, non-alphabetic ids)
        \end{itemize}
\end{itemize}

\textit{Example}: An embedded DSL for integer sets:\\
\begin{Haskell}
type IntegerSet = 
    -- constructors:
    empty :: IntegerSet
    insert :: Integer -> IntegerSet -> IntegerSet
    delete :: Integer -> IntegerSet -> IntegerSet
    -- observer:
    member :: Integer -> IntegerSet -> Bool

member 3 (insert 1 (delete 3 (insert 2 (insert 3 empty))))
    @$\equiv$@ False
\end{Haskell}

\textbf{(1)}
DSL as library of functions, implementation details exposed. 

\Haskellcode{../material/library-exposed.hs}


\subsection{Modules}

\begin{itemize}
    \item Group of related definitions (values, types) in a single file \\ (named ''M.hs'' / ''M.lhs''):\\
    \begin{Haskell}
module M where
    type Predicate a = a -> Bool
    id :: a -> a
    id x = x
    \end{Haskell}
    \item Hierarchy: module A.B.C.M in file A/B/C/M.hs
    \item Access definitions in other module M: \codeline{import M}

    \newpage
    \item Explicit export lists hide all other definitions:\\
    \begin{Haskell}
module M (id) where
    ...
    -- type Predicate a not exported
    \end{Haskell}
    \item \textbf{Abstract data types}:\\
    export algebraic data types, but \underline{not} its constructors:\\
        \begin{Haskell}
module M (Rose, leaf) where
    data Rose a = Node a [Rose a]
    leaf :: a -> Rose a [Rose a]
    leaf x = Node x []
\end{Haskell}
        
    \begin{itemize}
        \item Export constructors:\\
        \begin{Haskell}
module M (Rose(Node), leaf) where
    @\dots@
-- or export all constructors:
module M (Rose(..), leaf) where
        \end{Haskell}
        \item Instance definitions (including deriving) are exported with their type.
    \end{itemize}
    \item Qualified imports to partition name space:\\
    \begin{Haskell}
import qualified M
    ...
    -- use M.foobar syntax
    t :: M.Rose Char
    t = M.leaf 'x'
    \end{Haskell}
    \item Partially import module:\\
    \begin{Haskell}
-- only import nub and reverse
import Data.List (nub, reverse)

-- import whole module but without otherwise
import Prelude hiding (otherwise)
otherwise :: Bool
otherwise = False -- harhar
    \end{Haskell}
    \item Pass imported modules to importer of own module:\\
    \begin{Haskell}
module M (..., module Data.List, ...) where
    import Data.List (nub)
    \end{Haskell}
    \begin{Haskell}
import qualified M
    M.nub
    \end{Haskell}
\end{itemize}

% \codefile{haskell}{caption={SetLanguage.hs}, label=SetLanguage.hs}{../material/SetLanguage.hs}
% \codefile{haskell}{caption={SetLanguageShallow.hs}, label=SetLanguageShallow.hs}{../material/SetLanguageShallow.hs}
\Haskellcode{../material/SetLanguageShallowCard.hs}
\Haskellcode{../material/set-language-shallow.hs}

\textbf{(2)}
DSL as library of functions, abstract data type (module).
\begin{itemize}
    \item \textbf{Shallow DSL embedding}:\\
    Semantics of DSL operations directly expressed in terms of host language value (e.g. list or characteristic function)
    \begin{itemize}
        \item constructors (empty, insert, delete) perform the work, harder to add
        \item observers (member) trivial
    \end{itemize}
    \item \textbf{Deep DSL embedding}:\\
    DSL operations build an abstract syntax tree (AST) that represents applications and arguments
    \begin{itemize}
        \item constructors merely build the AST, very easy to add
        \item observers interpret (traverse) the AST and perform the work
    \end{itemize}
    % \codefile{haskell}{caption={SetLanguageDeep.hs}, label=SetLanguageDeep.hs}{../material/SetLanguageDeep.hs}
    \Haskellcode{../material/SetLanguageDeepCard.hs}
    \Haskellcode{../material/set-language-deep.hs}

\end{itemize}

\Haskellcode{../material/ExprDeepNum.hs}
\Haskellcode{../material/expr-deep-num.hs}

% \codefile{haskell}{caption={expr-language.hs}, label={expr-language.hs}}{../material/expr-language.hs}


% \codefile{haskell}{caption={ExprDeep.hs}, label={ExprDeep.hs}}{../material/ExprDeep.hs}
% \codefile{haskell}{caption={ExprDeepGADTUntyped.hs}, label={ExprDeepGADTUntyped.hs}}{../material/ExprDeepGADTUntyped.hs}


\subsection{Generalized Algebraic Data Types (GADTs)}

\textit{Idea}: 
\begin{itemize}
    \item Encode the type of a DSL expression (here: Integer or Bool) in its \textbf{Haskell representation type}.
    \item Use Haskell's type checker to ensure at compile time that only well-typed DSL expressions are built.
\end{itemize}

\textit{Language extensions:}\\
\codeline{\{-\# LANGUAGE GADTs \#-\}}
\begin{itemize}
    \item Define new parameterized type T, its constructors k$_i$ and their type signatures:\\
    \begin{Haskell}
data T a@$_1$@ a@$_2$@ @\dots@ a@$_n$@ where
    k@$_1$@ :: b@$_{11}$@ -> @\dots@ -> b@$_1n_1$@ -> T t@$_{11}$@ t@$_{1n}$@
    @\dots@
    k@$_r$@ :: b@$_{r1}$@ -> @\dots@ -> b@$_rn_r$@ -> T t@$_{r1}$@ t@$_{rn}$@
    \end{Haskell}
\end{itemize}

\Haskellcode{../material/ExprDeepGADTTyped.hs}
\Haskellcode{../material/expr-language-typed.hs}

% \codefile{haskell}{caption={ExprShallow.hs}, label={ExprShallow.hs}}{../material/ExprShallow.hs}
% \codefile{haskell}{caption={expr-language-shallow.hs}, label={expr-language-shallow.hs}}{../material/expr-language-shallow.hs}
%\Haskellcode{../material/ExprShallowPrint.hs}
%\Haskellcode{../material/expr-language-shallow-print.hs}



\subsection{Shallow embedding of a String Matching DSL}

\begin{itemize}
    \item \textbf{Pattern}:
    \begin{enumerate}
        \item Given a String, a pattern returns the list of matches. Match failure? Returns the empty list.
        \item A match consists of a value (e.g. the matched characters, tokens, parse trees) and the residual String left to match.
    \end{enumerate}
    Thus: \codeline{type Pattern a = String -> [(a, String)]}
\end{itemize}

\subsubsection{DSL design}
\vspace{9pt}\begin{center}\begin{tabular}{|c|c|c|}\hline
\rowcolor{grau} Pattern                 &               & DSL function     \\\hline
                match lit. char         & "x"           & \codeline{lit :: Char -> Pattern Char}    \\\hline
                match empty string      & $\epsilon$    & \codeline{empty :: a -> Pattern a}        \\\hline
                fail always             & $\emptyset$   & \codeline{fail :: Pattern a}              \\\hline
                alternative             & |             & \parbox[t]{9cm}{\codeline{alt :: Pattern a -> Pattern a -> Pattern a}}  \\\hline
                sequence                & .             & \parbox[t]{9cm}{\codeline{seq :: (a -> b -> c) ->} \\ \codeline{Pattern a -> Pattern b -> Pattern c}} \\\hline
                repetition              & *             & \codeline{rep :: Pattern a -> Pattern [a]} \\\hline
\end{tabular}\end{center}\vspace{9pt}

% \codefile{haskell}{caption={PatternMatch.hs}, label={PatternMatch.hs}}{./PatternMatch.hs}
% \codefile{haskell}{caption={pattern-match.hs}, label={pattern-match.hs}}{./pattern-match.hs}
%\Haskellcode{../material/PatternMatching.hs}

\begin{mdframed}[linecolor=black, topline=false, bottomline=false,
  leftline=false, rightline=false, backgroundcolor=mintedbackground]
    \inputminted[fontfamily=tt, breaklines]{Haskell}{../material/PatternMatching.hs}
\end{mdframed}

\Haskellcode{../material/pattern-match.hs}

\begin{mdframed}[linecolor=black, topline=false, bottomline=false,
  leftline=false, rightline=false, backgroundcolor=mintedbackground]
    \inputminted[fontfamily=tt, breaklines]{Haskell}{../material/pattern-matching.hs}
\end{mdframed}

%\Haskellcode{../material/pattern-matching.hs}

\pagebreak
%!TEX root = fp.tex

\section{Lazy Evaluation}
To execute a program, Haskell reduces expressions to values. Haskell uses normal order reduction to select the next expression to reduce:

\begin{enumerate}
   \item The outermost reducible expression (redex) is reduced first
   \item $\Rightarrow$ Function applications are reduced first before their arguments.
   \item If no further redex is found, the expression is in normal form and reduction terminates.
\end{enumerate}

\begin{Haskell}
fst :: (a, b) -> a
fst (x, y) = x
sqr :: Num a => a -> a
sqr x = x * x

-- ->: reduces to
fst (sqr (1+3), sqr 2) -> sqr (1 + 3) [fst]
                       -> (1+3) * (1+3) [sqr]
                       -> 4 * 4 [+/+]
                       -> 16 [*]
\end{Haskell}

Haskell avoids the duplication of work through graph reduction. Expression are shared (referenced more than once) instead of duplicated.\\
Lazy evaluation: normal order reduction and sharing.

\subsection{WHNF}
An expression e is in weak head normal form (WHNF) if it is of the following form
\begin{enumerate}
   \item v (where v is an atomic value \mint{Haskell}{Integer, Bool, Char, ...})
   \item \mint{Haskell}{c e1 e2 ... en} (where c is an n-ary constructor, like \mint{Haskell}{(:)})
   \item \mint{Haskell}{f e1 e2 ... em} (where f is an n-ary function, $m<n$)
\end{enumerate}

Haskell reduces values to WHNF only (stop criterium for reduction) unless we request reduction to normal form (e.g. when printing results).\\

Example expressions in WHNF:
\begin{itemize}
   \item \mint{Haskell}{42 --1.}
   \item \mint{Haskell}{(sqr 2, sqr 4) --2. (,)}
   \item \mint{Haskell}{f x : map f xs --2. (:)}
   \item \mint{Haskell}{Just (40+2) --2. Just}
   \item \mint{Haskell}{(* (40+2)) --3. * binär}
   \item \mint{Haskell}{(\x -> 40+2) --3. unary function w/o args}
\end{itemize}

\subsection{Lazy Evaluation and Bottom}
Some Haskell expressions have the value bottom. Examples: \mint{Haskell}{error, undefined, bomb}. Lazy evaluation admits functions that return a non-bottom value even if they receive bottom as argument (also: non-strict functions).
N-ary function f is strict in its i-th argument, if \mint{Haskell}{f x1 .. xi-1 bottom xi+1 ... xn = bottom}
Examples:
\begin{itemize}
   \item \mint{Haskell}{const :: a -> b -> a --strict in first, non-strict in second argument}
   \item \mint{Haskell}{(&&) :: Bool -> Bool -> Bool -- dito}
\end{itemize}
If a function pattern matches on an argument, Haskell semantics define it to be strict in that argument.\\
Example:\\
\begin{Haskell}
data T = T Int
f :: T -> Int
f (T x) = 42

f undefined -> undefined
f (T undefined) -> 42
\end{Haskell}


%14.01.2016

\section{Infinite Lists (Data Structures)}
One consequence of lazy evaluation: programs can handle infinite Lists as long as any run will inspect only a finite prefix of such a list.
Enables a modular programming style:
\begin{enumerate}
   \item generator functions produce an infinite number of solutions / approximations / \dots
   \item test functions select one (or finite number of) solutions from this infinite list.
\end{enumerate}

\subsection{Example: Newton-Raphson square root approximation}
Iteratively approximate the square root of x:
\begin{enumerate}
   \item $a_0 = \tfrac{x}{2}$
   \item $a_{i+1} = \tfrac{a_i+\tfrac{x}{a_i}}{2}$
\end{enumerate}

\Haskellcode{../material/newton-raphson.hs}

%\begin{Haskell}
%--1. Generator:
%iterate :: (a -> a) -> a -> [a]
%--iterate f x @\equiv@ [x, f x, f (f x), @\dots@]
%iterate f x = x : iterate x (f x)
%
%--2. Test:
%within :: (Ord a, Num a) => a -> [a] -> a
%--within eps xs : consumes xs until adjacent elements differ less than eps (for the first time)
%within eps (@x_1@:@x_2@:xs) | abs (@x_1@-@x_2@) <= eps = @x_2@
%                                                | otherwise                                = within eps (@x_2@:xs)
%\end{Haskell}

\subsection{Example: Tic-Tac-Toe game tree}
Build the (potentially huge) tree of possible moves for the Tic-Tac-Toe board game. Evaluate promise of game position. 
Plan:
\begin{enumerate}
   \item Find representation of game position (board + player next up)
\begin{verbatim}
|1|2|3| next: x
|4|5|6| square #6: open 
|0|x|0| spare #9: occupied by player 0
\end{verbatim}
   \item Provide pretty-printing for game positions.
   \item Define initial position and possible moves: \mint{Haskell}{moves :: Position -> [Position]}
   \item Evaluate a given position: \mint{Haskell}{static :: Position -> Int} (1 / -1: x/0 won the game, 0 draw)
   \item Build a game tree of positions: \mint{Haskell}{gameTree :: Position -> Tree Position}
   \item Rather than simple static evaluation, now evaluate positions based on possible game futures. $\Rightarrow$ in game tree, perform evaluation bottom up.
   \item Optimization ($\alpha-\beta$ algorithm)
\end{enumerate}

Code: tic-tac-toe.hs
%\Haskellcode{../material/tic-tac-toe.hs}

\pagebreak
%!TEX root = fp.tex

\section{Functors}
Type class Functor embodies the application of a function to the elements (or: inside) of a structure, which leaving structure (or: outside) alone.

\subsection{Examples}
\begin{Haskell}
map :: (a -> b) -> [a] -> [b]
mapTree :: (a -> b) -> Tree a -> Tree b

--Note: f is a type constructor that receives exactly one argument
--(Functor is also called a constructor class).
class Functor f where
   fmap :: (a -> b) -> f a -> f b
   
--Examples:
instance Functor [] where
   fmap = map
   
instance Functor Tree where
   fmap = mapTree
   
instance Functor Maybe where
   fmap f (Just x) = Just (f x)
   fmap f Nothing  = Nothing
\end{Haskell}

\begin{minipage}{\textwidth}
Type constructors can be partially applied. Uses prefix notation:\\
\begin{Haskell}
a -> b @$\equiv$@ (->) a b
(a, b) @$\equiv$@ (,) a b
[a] @$\equiv$@ [] a

--Examples (defines type constructors):
type Flagged = (,) Bool
type Indexed (->) Int
--MayFail e a: computation may yield value a or fail with error e
type MayFail e = Either e

instance Functor (Either e) where
   fmap f (Left e)  = Left e
   fmap f (Right x) = Right (f x)
   
instance Functor Flagged where
--fmap :: (a -> b) -> (Bool, a) -> (Bool, b)
   fmap f (b, x) = (b, f x)
   
instance Functor Indexed where
   fmap f g = f . g

--Functor Laws
fmap id @$\equiv$@ id
fmap (f . g) = fmap f . fmap g
\end{Haskell}
\end{minipage}
%\Haskellcode{../material/round.hs}

\subsection{Kinds - Types for Types}

\begin{center}\begin{tabular}{|c|c|c|}\hline
\rowcolor{grau} kind           & describes...                   & example\\\hline
\codeline{*}                 & types                             & \codeline{Int, Bool, (Bool, Int), [Char], ...} \\\hline
\codeline{* -> *}         & unary type constructor  & \codeline{Maybe,  []}\\\hline
\codeline{* -> * -> *} & binary type constructor & \codeline{Either, (,), (->)}\\\hline
\end{tabular}\end{center}


\section{Applicative}
\begin{Haskell}
--Compare:
(@$\$$@) :: (a -> b) -> a -> b
fmap :: Functor f => (a -> b) -> f a -> f b --<@$\$$@>
(<*>) :: Applicative f => f (a -> b) -> f a -> f b
\end{Haskell}

Read \codeline{<*>} as (ap)ply, ''tiefighter''.

\begin{Haskell}
class Functor f => Applicative f where
   pure :: a -> f a
   (<*>) :: f (a -> b) -> f a -> f b
   
--Make any Applicative f a Functor:
fmap g x = pure g <*> x
\end{Haskell}

Applicative embodies
\begin{enumerate}
\item function \underline{application} on the level of (contained) values.
\item \underline{combination} on the level of structures.
\end{enumerate}

\subsection{Applicative instances}
\begin{Haskell}
instance Applicative Maybe where
   pure x = Just x
   Just f <*> Just x = Just (f x)
   _ <*> _ = Nothing
   
instance Monoid c => Applicative ((,) c) where
   pure x = (mempty, x)
   (@$c_1$@, f) <*> (@$c_2$@, x) = (@$c_1$@ `mappend`@$c_2$@, f x)
   
instance Applicative [] where
   pure x = [x]
   fs <*> xs = [f x | f <- fs, x <- xs]
\end{Haskell}

\section{Interlude: Monoid}

Type class Monoid a represents combinable values of type a:

\begin{Haskell}
class Monoid a where
   mempty :: a -- empty, neutral element for mappend
   mappend :: a -> a -> a -- combination
   mconcat :: [a] -> a
\end{Haskell}

Examples: $(0, +), (1, \cdot), (\text{true}, \land), (\text{false}, \lor), (\{\}, \cup), ([], (++))$

\subsection{Monoid Laws}
\begin{Haskell}
mempty `mappend`xs @$\equiv$@ xs
xs `mappend`(ya `mappend`zs) @$\equiv$@ (xs `mappend`ys) `mappend`zs``
mconcat xs @$\equiv$@ foldr mempty mappend xs
\end{Haskell}
%
% \input{chapter07-monads.tex}
%
% \input{chapter08-parallelism.tex}


\end{document}
