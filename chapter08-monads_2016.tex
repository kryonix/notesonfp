%!TEX root = fp.tex

\section{Sequencing Functions}
\begin{Haskell}
(.) :: (b -> c) -> (a -> b) -> (a -> c)
f . g = @$\backslash$@x -> f (g x)

(>=>) :: (a -> b) -> (b -> c) -> (a -> c)
f >=> g = g . f

f1 >=> f2 >=> ... >=> fn -- composes the fi in left-to-right order
-- (>=>, id) forms a monoid.
\end{Haskell}

\subsection{Sequencing partial functions (a -> Maybe b)}
Return \codeline{Nothing} as soon as first function in pipeline returns \codeline{Nothing}

\begin{Haskell}
(>=>) :: (a -> Maybe b) -> (b -> Maybe c) -> (a -> Maybe c)
f >=> g = @$\backslash$@x -> case f x of
                                       Nothing -> Nothing
                                       Just y -> g y
\end{Haskell}

\subsection{Sequencing exception-generating functions (a -> Exc b)}
\begin{Haskell}
type Exc b = Either Error b
type Error = String
\end{Haskell}

Exceptions are propagated once occurred:

\begin{Haskell}
(>=>) :: (a -> Exc b) -> (b -> Exc c) -> (a -> Exc c)
f >=> g = @$\backslash$@x -> case f x of
                                       Left err -> Left err
                                       Right y -> g y
\end{Haskell}

\subsection{Sequencing non-deterministic functions}
\begin{Haskell}
(a -> NonDet b)
--Non-determinism: any answer in a list of answers:
type NonDet b = [b]
--Take all possible answers into account:
(>=>) :: (a -> NonDet b) -> (b -> NonDet c) -> (a -> NonDet c)
f >=> g = @$\backslash$@x -> concat [g y | y <- f x]
\end{Haskell}

\subsection{Sequencing ``stateful'' functions}
\begin{Haskell}
a -> State -> (State, b)
--State Transformer:
type ST b = State -> (State, b)
--Stateful function: a -> ST b
(>=>) :: (a -> ST b) -> (b -> ST c) -> (a -> ST c)
f >=> g = @$\backslash$@ x s0 -> let (s1, y) = f x s0
                                       let (s2, z) = g y s1 -- <=> in g y s1
                                       in (s2, z)
\end{Haskell}

\subsection{Sequencing side-effecting functions}
\begin{Haskell}
a -> World -> (World, b)
--Side effects: functions consume current world
--return new world along with result:
type World = ... -- abstract (defined by Haskell runtime)
type IO b = World -> (World, b)
--Value of type IO b: an action that
--1. when performed has a side-effect on the world, and then
--2. returns a value of type b.
\end{Haskell}

Haskell build-ins:
\begin{itemize}
	\item \codeline{print :: Show a => a -> IO ()}
	\item \codeline{putStrLn :: String -> IO ()}
	\item \codeline{getLine :: IO String}
	\item \codeline{readFile :: FilePath -> IO String}
\end{itemize}

\section{Type class Monad}
\begin{Haskell}
class Applicative m => Monad m where
   return :: a -> m a
   (>>=) :: m a -> (a -> m b) -> m b -- bind
\end{Haskell}

\subsection{Lifting}

\begin{center}\begin{tabular}{|c|c|}\hline
\rowcolor{grau} Monad & Lifting (return x) \\ \hline
Maybe & \codeline{Just x}\\
Exc & \codeline{Right x}\\
NonDet & \codeline{[x]}\\
ST & \codeline{|$\backslash$|s -> (s , x)} \\ \hline
\end{tabular}\end{center}

\begin{Haskell}
--Make Maybe an instance of Monad:
instance Monad Maybe where
   return = Just
   Nothing >>= g = Nothing
   Just y >>= g = g y

--Kleisli composition
(>=>) :: Monad m => (a -> m b) -> (b -> m c) -> (a -> m c)
f >=> g = @$\backslash$@x -> f x >>= g
\end{Haskell}

Let m be a Monad. Then m is also an Applicative:

\begin{Haskell}
pure = return
u <*> v = u >>= @$\backslash$@f -> v >>= @$\backslash$@x -> return (f x)
\end{Haskell}

\begin{Haskell}
(@\$@) :: (a -> b) -> a -> b
(<@\$@>) :: Functor m => (a -> b) -> m a -> m b
(<*>) :: Applicative m => m (a -> b) -> m a -> m b
(=<<) :: Monad m => (a -> m b) -> m a -> m b -- (=<<) = flip (>>=)
\end{Haskell}

\subsection{do-Notation}
\begin{Haskell}
--(e : expression of type m a, es: sequence of ;-separated expression):
do { e } @$\equiv$@ e
do {x <- e; es} @$\equiv$@ e >>= @$\backslash$@x -> do {es}
do {e; es} @$\equiv$@ e >>= @$\backslash$@_ -> do {es}
do {let v = x; es} = let v = x in do {es}
\end{Haskell}